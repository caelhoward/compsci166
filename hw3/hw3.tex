\documentclass[11pt]{exam}
\usepackage{amsmath}
\usepackage{amsthm}
\usepackage{amssymb}
\newcommand{\myname}{Cael Howard} %Write your name in here
\newcommand{\myUCO}{cthoward} 
\newcommand{\myhwtype}{Homework}
\newcommand{\myhwnum}{3} %Homework set number
\newcommand{\myclass}{Compsci 166}
\newcommand{\mylecture}{}
\newcommand{\mysection}{}

% Prefix for numedquestion's
\newcommand{\questiontype}{Question}


% Use this if your "written" questions are all under one section
% For example, if the homework handout has Section 5: Written Questions
% and all questions are 5.1, 5.2, 5.3, etc. set this to 5
% Use for 0 no prefix. Redefine as needed per-question.
\newcommand{\writtensection}{0}

\usepackage{amsmath, amsfonts, amsthm, amssymb}  % Some math symbols
\usepackage{enumerate}
\usepackage{enumitem}
\usepackage{graphicx}
\usepackage{hyperref}
\usepackage[all]{xy}
\usepackage{wrapfig}
\usepackage{fancyvrb}
\usepackage[T1]{fontenc}
\usepackage{listings}
\usepackage{accents}
\usepackage{braket}
\usepackage{tikz}
\usepackage{quantikz}
\usepackage{commath}

\usepackage{centernot}
\usepackage{mathtools}
\DeclarePairedDelimiter{\ceil}{\lceil}{\rceil}
\DeclarePairedDelimiter{\floor}{\lfloor}{\rfloor}
\DeclarePairedDelimiter{\card}{\vert}{\vert}


\setlength{\parindent}{0pt}
\setlength{\parskip}{5pt plus 1pt}
\pagestyle{empty}

\def\indented#1{\list{}{}\item[]}
\let\indented=\endlist

\newcounter{questionCounter}
\newcounter{partCounter}[questionCounter]

\newenvironment{namedquestion}[1][\arabic{questionCounter}]{%
    \addtocounter{questionCounter}{1}%
    \setcounter{partCounter}{0}%
    \vspace{.2in}%
        \noindent{\bf #1}%
    \vspace{0.3em} \hrule \vspace{.1in}%
}{}

\newenvironment{numedquestion}[0]{%
	\stepcounter{questionCounter}%
    \vspace{.2in}%
        \ifx\writtensection\undefined
        \noindent{\bf \questiontype \; \arabic{questionCounter}. }%
        \else
          \if\writtensection0
          \noindent{\bf \questiontype \; \arabic{questionCounter}. }%
          \else
          \noindent{\bf \questiontype \; \writtensection.\arabic{questionCounter} }%
        \fi
    \vspace{0.3em} \hrule \vspace{.1in}%
}{}

\newenvironment{alphaparts}[0]{%
  \begin{enumerate}[label=\textbf{(\alph*)}]
}{\end{enumerate}}

\newenvironment{arabicparts}[0]{%
  \begin{enumerate}[label=\textbf{\arabic{questionCounter}.\arabic*})]
}{\end{enumerate}}

\newenvironment{questionpart}[0]{%
  \item
}{}

\newcommand{\bravec}[2]{\begin{bmatrix} #1 & #2 \end{bmatrix}}

\newcommand{\ketvec}[2]{\begin{bmatrix} #1 \\ #2 \end{bmatrix}}
\newcommand{\transgate}[4]{\begin{bmatrix} #1 & #2\\ #3 & #4 \end{bmatrix}}

\newcommand{\hgate}[0]{$\transgate{1/\sqrt{2}}{1/\sqrt{2}}{1/\sqrt{2}}{-1/\sqrt{2}}$}
\newcommand{\xgate}{$\transgate{0}{1}{1}{0}$}
\newcommand{\cgate}{$\transgate{1}{0}{0}{i}$}
\newcommand{\zgate}{$\transgate{1}{0}{0}{-1}$}

\newcommand{\answerbox}[1]{
\begin{framed}
\vspace{#1}
\end{framed}}

\pagestyle{head}

\headrule
\header{\textbf{\myclass\ \mylecture\mysection}}%
{\textbf{\myname\ (\myUCO)}}%
{\textbf{\myhwtype\ \myhwnum}}

\begin{document}
\thispagestyle{plain}
\begin{center}
  {\Large \myclass{} \myhwtype{} \myhwnum} \\
  \myname{} (\myUCO{}) \\
  \today
\end{center}


%Here you can enter answers to homework questions

\begin{numedquestion}
    \textbf{1)} Design a two quibit circuit that starts in the state $\ket{00}$ and ends with $\ket{\Psi^-}$.
    \begin{center}
        \begin{quantikz}
            \ket{0}&\gate{X}&\gate{H}&\ctrl{1}&\\
            \ket{0}&&&\targ{}&\\
        \end{quantikz}
    \end{center}
    \textbf{2)} Write down $\ket{0}$ and $\ket{1}$ as a weighted sum of $\ket{\pi/6}$ and $\ket{4\pi/6}$ in the above basis.
    \begin{align*}
    \ket{0} &= \frac{\sqrt{3}}{2}\ket{\pi/6} - \frac{1}{2}\ket{4\pi/6}\\
    \ket{1} &= \frac{1}{2}\ket{\pi/6} + \frac{\sqrt{3}}{2}\ket{4\pi/6}
    \end{align*}
    \textbf{3)} If Alice measures her quibit in the standard basis, what are the probabilities of each outcome, and the state of the two quibits after the measurement?
    \begin{align*}
      \ket{0}: 1/2 \text{ Probability and state collapses to }\ket{01}\\
      \ket{1}: 1/2 \text{ Probability and state collapses to }\ket{10}
    \end{align*}

    \textbf{4)} If Alice instead chooses to measure in the $\{\ket{\pi/6}, \ket{4\pi/6}\}$ basis, what are the porbailities of each outcome, and the state of the two quibits after the measurement?
    \begin{align*}
    \ket{\Psi^-} &= \frac{1}{\sqrt{2}}((\frac{\sqrt{3}}{2}\ket{\pi/6} - \frac{1}{2}\ket{4\pi/6}) \otimes \ket{1}- (\frac{1}{2}\ket{\pi/6} + \frac{\sqrt{3}}{2}\ket{4\pi/6})\otimes \ket{0})\\
    \end{align*}
    Alice measures:
    \begin{center}
    $\ket{\pi/6}$ w/ probability 1/2, collapses to $\frac{\sqrt{3}}{2}\ket{\pi/6}\ket{1} - \frac{1}{2}\ket{\pi/6}\ket{0}$\\
    $\ket{4\pi/6}$ w/ probability 1/2, collapses to $\frac{-1}{2}\ket{4\pi/6}\ket{1} - \frac{\sqrt{3}}{2}\ket{4\pi/6}\ket{0}$
    \end{center}
    \textbf{5)} Verbally describe what happens to the second quibit when the first quibit of a $\ket{\Psi}$ state gets measures.
    
    \textbf{Answer:}\\
    The second quibit collapses into a state of $R_{-\theta}\ket{0}$ for $\theta = $ the state of the first quibit.
\end{numedquestion}
\newpage
\begin{numedquestion}
  \textbf{1)} Determine whether a state $\ket{\psi}$ that we know to be one of the two states is clonable or not. Briefly justify your answer.

  \textbf{Answer:}\\
  Yes. Since $\ket{i}$ and $\ket{-i}$ form an orthonormal basis, there must exist a $U$ that can clone a quibit that we know to be in one of the two above states.
  \vspace{1em}

  \textbf{2)} Design a quantum circuit that clones a state if we know it is in the above basis.
  \begin{center}
    \begin{quantikz}
    \ket{\psi} &\gate{S}&\gate{H}&\ctrl{1}&\gate{H}&\gate{S}&\\
    \ket{\ket{0}} &&& \targ{} & \gate{H} & \gate{S}&
    \end{quantikz}
  \end{center}

  \textbf{3)} Design a 4 quibit quantum circuit that clones 2 quibit standard basis states.
  \begin{center}
    \begin{quantikz}
    \ket{\psi_1}& \ctrl{2} & \qw & \qw\\
    \ket{\psi_2}& \qw & \ctrl{2} & \qw\\
    \ket{\psi_3}& \targ{} & \qw & \qw\\
    \ket{\psi_4}& \qw & \targ{} & \qw
    \end{quantikz}
  \end{center}

  \textbf{4)} Design a 4 quibit quantum circuit that clones the Bell basis states.
  \begin{center}
    \begin{quantikz}
    \ket{\psi_1}& \ctrl{1} & \gate{H} & \ctrl{2} & \qw & \qw &\gate{H} & \ctrl{1} & \qw\\
    \ket{\psi_2}& \targ{} & \qw & \qw & \ctrl{2} & \qw & \qw & \targ{} & \qw\\
    \ket{\psi_3}& \qw & \qw &\targ{} & \qw & \qw & \gate{H} & \ctrl{1} & \qw\\
    \ket{\psi_4}& \qw & \qw &\qw & \targ{} & \qw & \qw & \targ{} & \qw
    \end{quantikz}
  \end{center}
\end{numedquestion}
\end{document}

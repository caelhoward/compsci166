\documentclass[11pt]{exam}
\usepackage{amsmath}
\usepackage{amsthm}
\usepackage{amssymb}
\newcommand{\myname}{Cael Howard} %Write your name in here
\newcommand{\myUCO}{cthoward} 
\newcommand{\myhwtype}{Homework}
\newcommand{\myhwnum}{1} %Homework set number
\newcommand{\myclass}{Compsci 166}
\newcommand{\mylecture}{}
\newcommand{\mysection}{}

% Prefix for numedquestion's
\newcommand{\questiontype}{Question}


% Use this if your "written" questions are all under one section
% For example, if the homework handout has Section 5: Written Questions
% and all questions are 5.1, 5.2, 5.3, etc. set this to 5
% Use for 0 no prefix. Redefine as needed per-question.
\newcommand{\writtensection}{0}

\usepackage{amsmath, amsfonts, amsthm, amssymb}  % Some math symbols
\usepackage{enumerate}
\usepackage{enumitem}
\usepackage{graphicx}
\usepackage{hyperref}
\usepackage[all]{xy}
\usepackage{wrapfig}
\usepackage{fancyvrb}
\usepackage[T1]{fontenc}
\usepackage{listings}
\usepackage{accents}

\usepackage{centernot}
\usepackage{mathtools}
\DeclarePairedDelimiter{\ceil}{\lceil}{\rceil}
\DeclarePairedDelimiter{\floor}{\lfloor}{\rfloor}
\DeclarePairedDelimiter{\card}{\vert}{\vert}


\setlength{\parindent}{0pt}
\setlength{\parskip}{5pt plus 1pt}
\pagestyle{empty}

\def\indented#1{\list{}{}\item[]}
\let\indented=\endlist

\newcounter{questionCounter}
\newcounter{partCounter}[questionCounter]

\newenvironment{namedquestion}[1][\arabic{questionCounter}]{%
    \addtocounter{questionCounter}{1}%
    \setcounter{partCounter}{0}%
    \vspace{.2in}%
        \noindent{\bf #1}%
    \vspace{0.3em} \hrule \vspace{.1in}%
}{}

\newenvironment{numedquestion}[0]{%
	\stepcounter{questionCounter}%
    \vspace{.2in}%
        \ifx\writtensection\undefined
        \noindent{\bf \questiontype \; \arabic{questionCounter}. }%
        \else
          \if\writtensection0
          \noindent{\bf \questiontype \; \arabic{questionCounter}. }%
          \else
          \noindent{\bf \questiontype \; \writtensection.\arabic{questionCounter} }%
        \fi
    \vspace{0.3em} \hrule \vspace{.1in}%
}{}

\newenvironment{alphaparts}[0]{%
  \begin{enumerate}[label=\textbf{(\alph*)}]
}{\end{enumerate}}

\newenvironment{arabicparts}[0]{%
  \begin{enumerate}[label=\textbf{\arabic{questionCounter}.\arabic*})]
}{\end{enumerate}}

\newenvironment{questionpart}[0]{%
  \item
}{}

\newcommand{\answerbox}[1]{
\begin{framed}
\vspace{#1}
\end{framed}}

\pagestyle{head}

\headrule
\header{\textbf{\myclass\ \mylecture\mysection}}%
{\textbf{\myname\ (\myUCO)}}%
{\textbf{\myhwtype\ \myhwnum}}

\begin{document}
\thispagestyle{plain}
\begin{center}
  {\Large \myclass{} \myhwtype{} \myhwnum} \\
  \myname{} (\myUCO{}) \\
  \today
\end{center}


%Here you can enter answers to homework questions

\begin{numedquestion}
    \textbf{Question Statement:}\\
    Frankie the frog lives in a pond with two lily pads, east and west. One day, she found two coins at the bottom of the pond and has placed one on each of the two lily pads. Every morning, Frankie flips the coin on the lily pad she spent her last day on, and jumps to the other lily pad if it lands heads. If the coin is tails, she stays on her lily pad for the day.

    The state space is E and W, corresponding to the lily pad Frankie spends her day on. We cannot assume that the coins are fair, as we do not know where they came from! They could be weighted in very different ways. Let’s call the probability that the east coin lands on heads to be $p$ and the probability that the west coin lands on heads to be $q$.
    \vspace{2em}

    \textbf{1.1)} On day one, Frankie is on the east lily pad. How can we express this fact using a probability vector?

    \vspace{1em}
    \textbf{Answer:}\\
    We will define east state space E and W as follows:
    \begin{align*}
        | E \rangle &= \begin{bmatrix}
                        1 \\
                        0
                        \end{bmatrix} \\
        | W \rangle &= \begin{bmatrix}
            0 \\
            1
            \end{bmatrix}
    \end{align*}
    Therefore, the probability vector representing the fact that Frankie is on the east lily pad is $\begin{bmatrix} 1 \\ 0 \end{bmatrix} \\$
    \vspace{2em}

    \textbf{1.2) } Write down the stochastic matrix that corresponds to Frankie's game.
    \vspace{1em}

    \textbf{Answer:}\\
    The stochastic matrix that corresponds to this game is as follows:
    \begin{center}
    $\begin{bmatrix}
    1-p & q \\
    p & 1-q
    \end{bmatrix}$
    \end{center}

\end{numedquestion}






\end{document}
